
\section{Installation et Utilisation}
Commencez par compiler le projet. En étant à la racine du répertoire, exécutez la commande suivante.
\begin{verbatim}
    > make all
\end{verbatim}
\subsection{Serveur}
Pour exécuter le serveur, lancer la commande suivante
\begin{verbatim}
    > bin/serveur
\end{verbatim}

Laissez le terminal ouvert tant que vous utilisé l'application. Les logs du serveur apparaîtront dans ce terminal.

\subsection{Client}

Pour lancer un client, entrez ceci dans un terminal à la racine du projet.

\begin{verbatim}
    > bin/client <adresse_du_serveur>
\end{verbatim}

Une fois le client ouvert choisissez un pseudo et vous voilà prêt à utiliser l'application de messagerie.